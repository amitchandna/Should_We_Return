\chapter{Personal Freedom and Individualism vs Collectivism}
Being in India with family, you will feel like you have had your wings clipped and you are unable to do what you have been conditioned to do since you were a child; be independent. From a young age growing up in the west we are not conditioned to be taken care of by others from a societal point of view. At home we are taught to respect elders, and take of what they want but society also encourages us to put our elders in a home. In the west we love our independence and as a result of growing up in this environment it becomes a part of us. Coming to India/Asia will inherently mean that you will have to give up a portion of this when with family. This is done to avoid drama which can be created quite quickly here in South Asia. Indepence as well as many other western ideals are found in bits and pieces here yet the mindset of a westernized asian will never quite fit into the true asian world after having grown up outside of it as far as some basic things are concerned. These things primarily boil down to values that are impressed upon us from a young age by the society within which we grow up. I think that this independence aspect is what might drive the differences between these two societies. This can be seen with the idea of an arranged marriage. We all groan as we know that there is a difficulty in explaining the idea of an arranged marriage to non-indians and why indians abroad tend to be fine with this or even joke about it and why white westerners hate this ideal. While I could elaborate(and will) for this chapter I digress.
