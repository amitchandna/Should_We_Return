\chapter{Fitting In}
I think that right away, whenever you come to India/South Asia you will feel like you fit in. Suddenly, you look like everyone else, the same celebrities you grew up seeing in movies at home are represented everywhere and people know who they are. Afterall the population of south asia is nearly a third of the worlds people. This will ring true of course if you do not harbour much of the damage cause by internalised racism that comes from the west. If you identify with this portion of yourself, for the first time you will feel as though there is no need to try and be different, as you belong here. Of all the places in the world, this is a place that your ancestors come from. The land here has grown your ancestors, and there will be no other place in the world that you can claim as being your land. Even if you grew up in the USA as I did, your claim to that land is non-existent as it belongs the native American population (sure you might have legal rights to it but it is not yours.) Suddenly you will smell food that you recognize, from fried foods to daal/chaval or dosa and sambar. This is the food and smells of your people and your place. The idea of you smelling weird or having to worry about having left your jacket in the kitchen while your mom was cooking doesn’t matter. Those smells are normal now. The faces of people you see fit with the communities that you saw growing up. You know how old someone is, and know when to be respectful and when not to be respectful. Respect for your elders is suddenly not just something you do behind closed doors/within a certain context of people. It is now the norm and it is weird to not respect your elders in public. Suddenly, your whole world that you grew up around closed doors is no longer behind closed doors. It is the world. As far as everyone is concerned there is no other way of life. You eat with your hands, you take time for food, you do not rush because there is no rush. All the performative aspects of culture that you have done in secret almost in the west are no longer performative in secret, they are normal and practiced across the board now without being hidden from you. The weather is nicer and you can use the sun to help you with your day(the things your family used to do in the west as far as tea times are concerned will suddenly make sense) The icing on the cake though is that when people ask you where you are from they wont settle for you being from the west. Your roots are here and they know that you are from here so when they ask you “where are you from”, you can actually say the state your family comes from and they will not only know where it is, they will also have something to tell you about the state for better or worse. They also will not settle for the state – they will want to know the city or village from which your family hails. They might even have friends from there as well. The language change may be strange, if you don’t speak an Indian language it will be a shock since everyone expects that you speak just like they do. This will take time to adjust to, but you will adjust to it and if you are quick to learn language you will be fine within a month or two. The most beautiful thing about these changes in environment are that you don’t have to do anything to feel accepted. For once it is because of the melanin in your skin. You were blessed with the ability to exist in this world as it is yours and your facial features and colour are the keys that put you into this world. While you will fit in, there will be a slight culture shock, it is expected no matter where you are. The culture shock I am referring to are habits such as those with mobile phones, conversation patterns, and general way of life within the confines of this world. It will be slightly different, but nothing that you cannot adapt to – you just have to try. Afterall our own family members adapted to the western way of life and succeeded, so the ability to adapt is in your genes. For the first time in your life, without really trying you will (almost) fully fit in. 
