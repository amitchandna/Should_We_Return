\chapter{Message to the reader}
	What is to follow is a collection of thoughts that I had while living in India over the course of 6 months as an adult for work – not something that each of us have had the privilege to do due to many different circumstances. My time in India was primarily for the completion of some field work for my Master’s thesis in Agricultural Development. This work is not a common field for an Indian-American/Canadian to get involved with for many reasons; however it was something that I am incredibly grateful for as it has shown me a completely different side of the world that we live in. As this experience will not translate to everyone’s experiences, I believe that sharing my experience may be worthwhile for others that find themselves in a similar boat in being caught between the east and the west as a young adult. To help put this entire writing in perspective I figure it would be good to get to know a little bit more about who I am and where I am coming from as an author of this text to see if I even slightly line up with a portion of your life. I grew up in Austin Texas throughout my childhood to a father of Punjabi descent and a mother of Bihari descent. Both of my parents however grew up in the Canada and the USA respectively and my grandparents on either side immigrated to the West in the 1960s in search of a better life compared against what was available in a newly independent South Asia. This makes me a 2nd or 3rd generation Indian-American (depending on how you define those terms). This means that for many Indians that are present in the USA, I am one generation further removed from south asia than you may be. This means that growing up I did not visit India to see my grandparents or aunts/uncles. I went to Pittsburgh, Pennsylvania and Windsor, Ontario for these things. We did go to India once as a kid for 2 weeks to see family there but this trip was incredibly brief and being a kid you do not get to see the Indian world in excess rather you get to see what is present in the homes of your family members with the main noticeable difference being language. Having grown up within this diasporic realm, I grew up with bits and pieces of the Indian world being represented in my daily life with other things not being represented as heavily. Things that I grew up with all the time that can be considered to be quite Indian were the easy things, like Indian food for dinner almost every day, but a peanut butter and jelly sandwich in my lunch box for school. We went to temple as a kid but then went less frequently as teenagers. Maintenance of Indian culture through Hinduism was present, but not nearly to the same extent as my peers whose parents had immigrated to the United States. I grew up speaking English at home and very rarely if ever ran into language issues with my grandparents as they all speak fluent English. As a second language I was taught Spanish in school and speak it currently and have had relationships(both friendships and romantic partners) where Spanish was our primary language. I only started learning Hindi after I had lived in Tanzania and was finding that it was useful to have a couple words here and there to interact with the Indian population. This brings me to the next point of interest. I grew up in the USA, but since then have lived in Canada, Tanzania, Denmark, and India. This has given me the benefit of getting to live in other contexts and see what it is like to be Indian across the world and how Western and Eastern societies function. My field of work to this point has been in Agricultural Development work to help address the issues of hunger that primarily plague communities of colour across the global south as these communities are going to pay the highest price for the effects of climate change while those of us in the global north get to skate by on our geographical location and different societal structures. As a child of the south Asian diaspora, I found that after joining different facebook groups and seeing that my life is still incredibly similar to that of other Indians across the globe, I started to wonder things like is it worth fighting the system all the time for a seat at the table especially when I may understand the problem. (Not all fields are like this, but the development field in agriculture is very much this way). This book/manifesto of sorts will not really dive into the concept of people of colour very much as this is not something I can do justice to and the amount of literature that is available for this is quite large. I am more focused on looking into the idea of whether or not it makes sense on a social side of things to have South Asians from abroad moving back towards south Asia to help increase the world from which we all come from. This will assume that you as a reader are well versed in the issues that exist as a person of colour in the west and will aim to answer the question of “but what is it like over there? And what about all the social issues that exist such as …..” Hopefully this will help you to see the east in a different light and help you to know if you want to work towards improving that world with the skill set you have or if you want to stay in the west. 
